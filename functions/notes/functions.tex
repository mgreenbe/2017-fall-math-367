\documentclass{amsart}

\newcommand{\lra}{\longrightarrow}
\newcommand{\RR}{\mathbb{R}}
\newcommand{\bx}{\mathbf{x}}
\newcommand{\bzero}{\mathbf{0}}
\renewcommand{\bf}{\mathbf{f}}

\begin{document}
\title{Vector-valued functions of a vector variable}
\maketitle

\begin{itemize}
\setlength\itemsep{1em}
\item Let $X$ and $Y$ be sets. A function
\[
  f:X\lra Y
\]
is a rule that associates a value $f(x)\in Y$ to every element $x\in X$.

\item Single variable calculus:
\[
 f:X\lra \RR,\quad\text{where}\quad
 X\subseteq\RR. 
\]
Typically, $X$ is an interval or a union of such.

\item Multivariable calculus:
\[
  \bf:X\lra \RR^m,\quad\text{where}\quad
X\subseteq \RR^n.
\]
Here,
\[
  \RR^n:=\{\bx=(x_1,\ldots,x_n) : x_1,\ldots,x_n\in\RR\}.
  \]

\item $\bf$ itself can be viewed as a vector:
\[
  f(\bx) = (f_1(\bx),\ldots,f_m(\bx)),
\]
where
\[
  f_i(\bx):=\text{$i$-th component of $f(\bx)$},\quad
  1\leq i\leq m.
  \]
Note that $f_i$ is a real-valued (scalar-valued)
function of the vector variable $\bx\in\RR^n$.
\[
  f_i:\RR^n\lra\RR.
  \]
We call $f_i$ the $i$-th component function of $\bf$ and we write
\[
\bf=(f_1,\ldots,f_m).
\]

\item Scalar-valued functions of a vector variable:

\bigskip
\begin{enumerate}
\setlength{\itemsep}{1em}
\item Define $f:\RR^2\lra \RR,\quad f(\bx) := x_1-x_2$.
\item Define $f:\RR^2\lra \RR,\quad f(x,y) := e^{-\pi(x^2+y^2)}$.
\item Define $f:\RR^n\setminus\{\bzero\}\lra \RR,\quad f(\bx) := \dfrac1{\|\bx\|^2}$.
($X\setminus Y:= \{x\in X : x\notin Y\}$)
\end{enumerate}

\item Natural domains: If $\bf$ is defined via formula without explicit reference to a domain
$X\subseteq \RR^n$, we take $X$ --- the natural domain of $\bf$ --- to be the largest subset
of $\RR^n$ on which the formula defining $\bf$ makes sense.

\bigskip
\begin{enumerate}
\setlength{\itemsep}{1em}
\item $g(t) = \left(\dfrac1{t^2-1},\dfrac1{t^3-1}\right)$, $X=\RR\setminus\{-1, 1\}$.
\item $f(x,y)=\log(y-x)$, $X=\{(x,y)\in\RR^2 : y>x\}$.
\item $\theta(x,y,z)=\cos^{-1}\left(\dfrac{z}{\sqrt{x^2+y^2+z^2}}\right)$, $X=\RR^3\setminus\{(0,0,0)\}$.
\end{enumerate}
\end{itemize}
\end{document}